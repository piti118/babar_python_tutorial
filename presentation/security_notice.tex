\documentclass{beamer}
\usepackage{hyperref}

% \usepackage{beamerthemesplit} // Activate for custom appearance
\begin{document}

\frame{
	\frametitle{Python Analysis Tool Tutorial}
	\begin{block}{By}
		Piti Ongmongkolkul and Chih-hsiang Cheng
	\end{block}
	\begin{block}{Our Websites}
	\small\url{http://piti118.github.com/babar_python_tutorial/}
	\end{block}
	\begin{block}{Hypernews}
	Python/IPython Analysis Tools\\
	\tiny\url{http://babar-hn.slac.stanford.edu:8080/HyperNews/get/pythonAnalTools.html}
	\end{block}
	\begin{block}{Agenda}
		\begin{itemize}
		\item Very Basic Python
		\item Reading root file and plotting them
		\item Multivariate Analysis
		\item Fitting
		\end{itemize}
	\end{block}
	
}

\frame
{
	\frametitle{Security Notice}
	\begin{block}{Short Version}
		{\color{red}DO NOT} use IPython notebook on shared computers (yakut noric etc.). The default setting is very unsecure.
	\end{block}
	\begin{block}{Long Version}
		{\color{red}DO NOT} use IPython notebook on shared computers unless you know what you are doing.
		\begin{itemize}
			\item The default setting only listen on localhost with no password. This is OK on your personal computer. But, on shared computer everyone is on localhost.
			\item IPython executes code as you. So, anyone with access to localhost will be able
			      to execute code as you. This means they can delete your files and edit your codes and a bunch of other things.
			\item It is possible to setup password and have it binds to non-localhost/specific port etc. But, I won't go into details but the keyword is \emph{IPython profile}.
		\end{itemize}
	\end{block}
}

\frame{
	\frametitle{Help us develop}
	\begin{block}{root\_numpy, iminuit, probfit are relatively young}
		If you are interested in helping. Look at their github pages.
		\begin{itemize}
			\item root\_numpy \url{https://github.com/rootpy/root_numpy}
			\item iminuit \url{https://github.com/iminuit/iminuit}
			\item probfit \url{https://github.com/iminuit/probfit}
		\end{itemize}
	\end{block}
	\begin{block}{Git is different.}
		You don't need to ask for write permission to repository. You clone repository do your magic and tell us ``hey, look at my amazing work" by making a pull request. Don't be shy.\\
		I often make pull request to matplotlib and IPython. It's fun.	
	\end{block}
	
}
\end{document}
